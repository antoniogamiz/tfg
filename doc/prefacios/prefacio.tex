\thispagestyle{empty}

\begin{center}
{\large\bfseries Any2Vec \\ Embeddings para búsqueda de semántica y aritmética de contenidos no verbales }\\
\end{center}
\begin{center}
Antonio Gámiz Delgado \\
\end{center}

%\vspace{0.7cm}

\vspace{0.5cm}
\noindent{\textbf{Palabras clave}: \textit{software libre}, \textit{embeddings}, \textit{propagación hacia atrás}, \textit{redes neuronales}, \textit{word2vec}, \textit{tropos}
\vspace{0.7cm}

\noindent{\textbf{Resumen}\\

Un tropo, como describe Rizzo Michael \cite{rizzo2013art}, es una imagen
universalmente reconocida, con varias capas de significado contextual que
crean un nueva metáfora visual. En el pasado se han hecho análisis de conjuntos
de tropos usando algoritmos evolutivos \cite{garcia2020startroper}. Sin embargo,
nunca se han usado técnicas del procesamiento de lenguaje natural, como \textit{Word2Vec}
para intentar extraer información adicional sobre tropos.

En este trabajo se desarrolla una extensión de \textit{Word2Vec}, que nos permite usar este modelo
para cualquier contenido no verbal, en particular, tropos. Además, de la misma forma que ocurre en \textit{Word2Vec},
se pueden realizar operaciones artitméticas (sumar y restar) con esos contenidos no verbales.

\cleardoublepage

\begin{center}
	{\large\bfseries Any2Vec \\ Embeddings for semantic search and non-verbal content arithmetic}\\
\end{center}
\begin{center}
	Antonio Gámiz Delgado\\
\end{center}
\vspace{0.5cm}
\noindent{\textbf{Keywords}: \textit{open source}, \textit{floss}, \textit{embeddings}, \textit{back-propagation}, \textit{neural networks}, \textit{word2vec}, \textit{tropes}
\vspace{0.7cm}

\noindent{\textbf{Abstract}\\

A trope, as Rizzo Michael \cite{rizzo2013art} describes, is a universally recognized image, with several layers of contextual meaning that create a new visual metaphor.
Trope set analyses have been done in the past using evolutionary algorithms \cite{garcia2020startroper}. However, natural language processing techniques such as \textit{Word2Vec}
have never been used to try to extract additional information about tropes.

In this work, an extension of \textit{Word2Vec} is developed, which allows us to use this model
for any non-verbal content, in particular, tropes. Also, just like in \textit{Word2Vec},
arithmetic operations (add and subtract) can be performed with these non-verbal contents.

\cleardoublepage

\thispagestyle{empty}

\noindent\rule[-1ex]{\textwidth}{2pt}\\[4.5ex]

D. \textbf{Juan Julián Merelo Guervós}, Profesor del Departamento de Arquitectura y Tecnología de Computadores y
D. \textbf{Serafín Moral Callejón}, Profesor del Departamento de Ciencia de Computadores e Inteligencia Artificial
\vspace{0.5cm}

\textbf{Informan:}

\vspace{0.5cm}

Que el presente trabajo, titulado \textit{\textbf{Any2Vec, Embeddings para búsqueda de semántica y aritmética de contenidos no verbales}},
ha sido realizado bajo mi supervisión por \textbf{Antonio Gámiz Delgado}, y autorizo la defensa de dicho trabajo ante el tribunal
que corresponda.

\vspace{0.5cm}

Y para que conste, expiden y firman el presente informe en Granada a Junio de 2022.

\vspace{1cm}

\textbf{El/la director(a)/es: }

\vspace{5cm}

\noindent \textbf{Juan Julián Merelo Guervós, Serafín Moral Callejón}

\chapter*{Agradecimientos}

A mis seres queridos.
