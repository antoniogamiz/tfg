\chapter{Implementación}

La implementación del software se ha dividido en hitos. Estos, han sido definidos en Github
y cada uno de ellos contiene un grupo de \textit{issues} que se corresponden con las distintas
mejoras que se han ido incorporando al software a lo largo de su desarrollo.

\section{Configuración inicial}

En este primer hito lo que se ha hecho es preparar las herramientas necesarias para poder desarrollar
el proyecto de forma cómoda. Esto incluye tareas como:

\begin{itemize}
  \item Detección de faltas: cada vez que se hace cualquier cambio en el texto del proyecto, se lanza un proceso
  automático que comprueba que cada palabra esté en un determinado diccionario (español en este caso).
  \item Definición de los objetivos y las historias de usuario: para poder empezar a desarrollar el trabajo es importante
  tener claro qué se va a hacer en él y en qué partes se va a dividir ese trabajo. Como se comentaba en la metodología, se
  ha seguido una metodología \textit{agile} en la que se intenta que cada nueva adición al trabajo aporte valor al usuario final
  (que en este caso es, principalmente, el tribunal).
  \item Compilación del proyecto: por razones evidentes se ha usado \textit{LaTex} para escribir el trabajo. El proceso de compilación
  requiere de configuración extra por lo que este proceso se ha automatizado de forma que:
  \begin{itemize}
    \item Cualquier persona que se descargue el código fuente pueda compilar el proyecto de forma cómoda.
    \item Cada vez que se integre una nueva parte al proyecto, éste se construya y publique automáticamente para facilidad de los tutores
    del trabajo que no puedan compilar el proyecto en su ordenador.
  \end{itemize}
\end{itemize}

