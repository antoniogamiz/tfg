\chapter{Implementación}

La implementación del software se ha dividido en hitos. Estos, han sido definidos en Github
y cada uno de ellos contiene un grupo de \textit{issues} que se corresponden con las distintas
mejoras que se han ido incorporando al software a lo largo de su desarrollo.

\section{Configuración inicial}

En este primer hito lo que se ha hecho es preparar las herramientas necesarias para poder desarrollar
el proyecto de forma cómoda. Esto incluye tareas como:

\begin{itemize}
  \item Detección de faltas: cada vez que se hace cualquier cambio en el texto del proyecto, se lanza un proceso
  automático que comprueba que cada palabra esté en un determinado diccionario (español en este caso).
  \item Definición de los objetivos y las historias de usuario: para poder empezar a desarrollar el trabajo es importante
  tener claro qué se va a hacer en él y en qué partes se va a dividir ese trabajo. Como se comentaba en la metodología, se
  ha seguido una metodología \textit{agile} en la que se intenta que cada nueva adición al trabajo aporte valor al usuario final
  (que en este caso es, principalmente, el tribunal).
  \item Compilación del proyecto: por razones evidentes se ha usado \textit{LaTex} para escribir el trabajo. El proceso de compilación
  requiere de configuración extra por lo que este proceso se ha automatizado de forma que:
  \begin{itemize}
    \item Cualquier persona que se descargue el código fuente pueda compilar el proyecto de forma cómoda.
    \item Cada vez que se integre una nueva parte al proyecto, éste se construya y publique automáticamente para facilidad de los tutores
    del trabajo que no puedan compilar el proyecto en su ordenador.
  \end{itemize}
\end{itemize}

Las herramientas que se han usado para implementar el trabajo son:

\begin{enumerate}
  \item Lenguaje de programación: se ha escogido \textit{Python} porque es un lenguaje muy famoso en el ámbito de la inteligencia artificial y el análisis de datos. Además
  tiene un coste de aprendizaje muy bajo, por lo que el código puede ser entendido fácilmente en poco tiempo (a diferencia de otras alternativas como \textit{C++}). \textit{Javascript}
  también es un lenguaje muy conocido, pero presenta muchas peculiaridades que hacen que sea difícil de entender en ocasiones. Por eso y por el hecho de que este trabajo no contiene ninguna
  interfaz web, también se ha descartado esa opción.
  \item Gestor de tareas: se ha escogido \textit{poetry} porque agrupa muchas de las funcionalidades necesarias para este trabajo. Como levantar entornos virtuales,
  gestión de dependencias y soporte para scripts. Hay otras alternativas como \textit{pip} (solo incluye instalación de paquetes) o \textit{conda} (que tiene flujos de trabajo poco intuitivos).
  \item Conjunto de datos: se ha optado por escoger los datos de \href{https://github.com/rhgarcia/tropescraper}{TropeScraper}, porque además de incluir conjuntos de datos listos para descargar, provee
  un paquete en Python para generar un conjunto de datos actualizados en cualquier momento.
\end{enumerate}

