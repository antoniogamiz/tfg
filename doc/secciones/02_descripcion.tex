\chapter{Descripción del problema}

En este capítulo se va a expandir el problema que brevemente se ha comentado en
la introducción, así como los objetivos a resolver por este trabajo.

\section{Problema a resolver}

A la hora de crear una obra cultural, como una película, los guionistas de cine
tienen que crear una trama argumental. De forma ineludible, esa trama va a
estar caracterizada por un conjunto de tropos. Es interesante destacar, que la
guionista no tiene por qué ser consciente de los tropos que está usando, o incluso podría estar creando tropos nuevos, que serán usados por otros
escritores.

Esa incertidumbre, puede incitar las siguientes preguntas: ¿qué conjunto de
tropos funcionan bien juntos?, ¿hay conjuntos de tropos que no casen bien?,
¿qué combinación de tropos puedo usar para aumentar la probabilidad de éxito de
mi película? Seguramente cada guionista se haga esas preguntas mientras decide
si bromear sobre Adolf Hitler \cite{tropo:AdolfHitlarious} o introducir a unos
alienígenas que abducen vacas \cite{tropo:AliensStealCattle}.

Por lo tanto, el problema a resolver es el siguiente: dado un conjunto de
tropos, encontrar subconjuntos de tropos cuyo \emph{rating} sea máximo.

\section{Objetivos} \label{section:goals}

Una vez está claro el problema a resolver, se pueden enumerar los principales
objetivos del trabajo:

\begin{enumerate}
      \item \label{obj:1} Crear una herramienta que permita hacer \textit{scraping} de los distintos
            tropos de \url{https://tvtropes.org/}. La herramienta debe ser fácilmente extensible a
            otras páginas de tropos. \href{https://github.com/antoniogamiz/tfg/milestone/3}{Milestone asociada}.
      \item \label{obj:2} Crear una herramienta que permita hacer \textit{scraping} de ratings de
            películas en \url{https://www.imdb.com/}. La herramienta debe ser fácilmente extensible a
            otras páginas de ratings.
      \item \label{obj:3} Recopilación y estudio teórico de los fundamentos matemáticos de las redes
            neuronales y del algoritmo \textit{word2vec}.
      \item \label{obj:4} Desarrollar interfaces para que los diferentes guionistas puedan consultar
            de forma sencilla qué rating estimado obtendrían a partir de los tropos de su obra. Es necesario
            que haya al menos dos interfaces: una fácil de usar por usuarios no técnicos y otra que sea
            fácilmente integrable con otros proyectos de código.
\end{enumerate}

Del primer objetivo \ref{obj:1}, se pueden refinar los siguientes casos de uso:

\begin{enumerate}
      \item \label{us:1} La herramienta debe ser altamente paralelizable. También debe de tener en
            cuenta a la página que se está atacando, es decir, no se debe de sobrecargar de peticiones.
      \item \label{us:2} Dado el tiempo que se necesita para hacer scraping, es necesario tener un
            sistema de pausa por si el usuario tiene que parar de hacer scraping o el proveedor está
            temporalmente caído.
      \item \label{us:3} La herramienta debe poder ser usada por la comunidad, es decir, debe ser
            publicada con instrucciones de uso e instalación. De forma que cualquier usuario pueda usarla.
      \item \label{us:4} El formato de salida de la herramienta tiene que ser configurable. Poder elegir
            varios formatos facilita su uso (afecta a \label{us:3})
\end{enumerate}