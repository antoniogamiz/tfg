\chapter{Descripción del problema}

En este capítulo se va a expandir el problema que brevemente se ha comentado en
la introducción, así como los objetivos a resolver por este trabajo.

\section{Problema a resolver}

A la hora de crear una obra cultural, como una película, los guionistas de cine
tienen que crear una trama argumental. De forma ineludible, esa trama va a
estar caracterizada por un conjunto de tropos. Es interesante destacar, que la
guionista no tiene por qué ser consciente de los tropos que está usando, o incluso podría estar creando tropos nuevos, que serán usados por otros
escritores.

Esa incertidumbre, puede incitar las siguientes preguntas: ¿qué conjunto de
tropos funcionan bien juntos?, ¿hay conjuntos de tropos que no casen bien?,
¿qué combinación de tropos puedo usar para aumentar la probabilidad de éxito de
mi película? Seguramente cada guionista se haga esas preguntas mientras decide
si bromear sobre Adolf Hitler \cite{tropo:AdolfHitlarious} o introducir a unos
alienígenas que abducen vacas \cite{tropo:AliensStealCattle}.

Por lo tanto, el problema a resolver es el siguiente: dado un conjunto de
tropos, encontrar subconjuntos de tropos cuyo \emph{rating} sea máximo.

\section{Objetivos} \label{section:goals}

Dado el problema anterior, los objetivos a resolver son:

\begin{enumerate}
      \item \label{obj:1} Crear una herramienta para que personas principalmente no técnicas, como
            guionistas, puedan modificar los tropos de las tramas argumentales de sus películas con confianza.
      \item \label{obj:2} Crear la base teórica, apartir de la de Word2Vec, para poder usar el algoritmo con conjuntos de tropos (más generalmente, conjuntos sin orden)
\end{enumerate}
