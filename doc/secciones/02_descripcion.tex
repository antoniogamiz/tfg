\chapter{Descripción del problema}

En este capítulo se va a expandir el problema que brevemente se ha comentado en
la introducción, así como los objetivos a resolver por este trabajo.

\section{Problema a resolver}

A la hora de crear una obra cultural, como una película, los guionistas de cine
tienen que crear una trama argumental. De forma ineludible, esa trama va a
estar caracterizada por un conjunto de tropos. Es interesante destacar, que la
guionista no tiene por qué ser consciente de los tropos que está usando, o incluso podría estar creando tropos nuevos, que serán usados por otros
escritores.

Esa incertidumbre, puede incitar las siguientes preguntas: ¿qué conjunto de
tropos funcionan bien juntos?, ¿hay conjuntos de tropos que no casen bien?,
¿qué combinación de tropos puedo usar para aumentar la probabilidad de éxito de
mi película?Seguramente, cada guionista se haga esas preguntas mientras decide
si bromear sobre Adolf Hitler \cite{tropo:AdolfHitlarious} o introducir a unos
alienígenas que abducen vacas \cite{tropo:AliensStealCattle}.

Por lo tanto, el problema a resolver es el siguiente: dado un conjunto de
tropos, encontrar subconjuntos de tropos cuyo \emph{rating} sea máximo.

\section{Objetivos} \label{section:goals}

Una vez está claro el problema a resolver, se pueden enumerar los principales
objetivos del trabajo:

\begin{enumerate}
    \item Desarrollar una herramienta útil para que los guionistas puedan
          consultar una estimación del rating de los tropos de su guión.
    \item Permitir que la funcionalidad de este trabajo pueda ser integrada con
          el mayor número de clientes.
    \item Crear un proyecto extensible y de código abierto para que futuros
          contribuidores al trabajo puedan complementar el trabajo con otras obras
          culturales como libros o cómics.
\end{enumerate}