\chapter{Introducción}

Un tropo, como describe Rizzo Michael \cite{rizzo2013art}, es una imagen
universalmente reconocida, con varias capas de significado contextural que
crean un nueva metáfora visual. Un buen ejemplo, es el tropo \emph{I can
    explain} \cite{tropo:ICanExplain}, donde un caracter que se siente culpable,
reacciona a la aparición de una figura autoritaria diciendo: \emph{"Puedo
    explicarlo!"}, procediendo a explicar lo que sea que haya pasado. Este tropo ha
sido enormemente utilizado en una infinidad de situaciones en obras culturales
\cite{tropo:ICanExplain}, desde \emph{Iron Man} hasta la serie de animación
\emph{Digimon}.

Cada película, libro o teatro, está formado por una inmensa cantidad de tropos
y, en cierta medida, su trama es caracterizada por ellos. Esa misma
caracterización, motiva un gran interés en el estudio de estos tropos: cómo se
relacionan entre ellos, cómo de comunes son, qué implicaciones tiene que una
obra cultural presente un cierto conjunto de tropos, etc.

Ese estudio es la principal motivación de este trabajo. Dado un conjunto de
obras culturales y los tropos que aparecen en ellas, se quiere estudiar la
relación existente entre los tropos y su rating, para obtener el máximo \emph
{rating} posible. En particular, este trabajo se centra en el estudio de tropos
encontrados en películas.


El trabajo está orgnizado en diferentes bloques:

\begin{enumerate}
    \item Bloque I: se describe en detalle el problema que se trata en el
          trabajo, seguido del estado del arte en el uso del NLP para encontrar
          relaciones entre palabras. A continuación, se detalla la planificación
          seguida para resolver el problema, así como las prácticas seguidas para
          realizarla.
    \item Bloque II: desarrollo teórico de la base matemática usada para la
          realización de este trabajo.
    \item Bloque III: se describe el proceso de análisis e implementación
          llevados a cabo, así como unas conclusiones finales.
\end{enumerate}

Este proyecto es software libre, y está liberado con la licencia \cite{gplv3}.