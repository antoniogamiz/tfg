\chapter{Planificación}

En este capítulo se detalla la planificación del proyecto y las metodologías y buenas prácticas
usadas para el desarrollo del mismo. Siguiendo uno los objetivos de este trabajo, la planificación
ha sido muy influenciada por el desarrollo ágil aplicado a la ciencia \cite{desarrolloagil}.

\section{Metodología utilizada}

\subsection{Kanban}

Como el objetivo del trabajo es \textit{ser ágil}, es necesario escoger una metodología de trabajo que
lo permita. En este proyecto se ha escogido \textit{Kanban}. Los objetivos principales de kanban son dar
visibilidad al trabajo que se está haciendo, maximizar la eficiencia y limitar la cantidad de trabajo
en progreso. Esto se hace a través de un tablero en el que de un simple vistazo, se puede consultar el
estado del trabajo.

Kanban se base en hacer pequeños cambios incrementales de forma que el esfuerzo para que se puede realizar
la tarea sea mínimo. Esto no implica que Kanban favorezca la rapidez sobre la calidad, al contrario,
prevalece la calidad del producto final sobre la velocidad de desarrollo.

\subsection{Desarrollo dirigido por pruebas}

Kanban es una metodología para organizar el trabajo. A la hora de escribir código, es necesario usar
buenas prácticas adicionales. El desarrollo dirigido por tests (TDD, del inglés \textit{Test Driven Development})
es un gran complemento a Kanban, ya que nos permite hacer incrementos pequeños que aporten valor al
producto final. El flujo de trabajo consiste en:

\begin{itemize}
    \item Escribir test: definir un test que comprueba que la nueva funcionalidad funciona como es debido.
          Evidentemente, al principio debe fallar (salir en rojo) porque no hay nada implementado.
    \item Implementar funcionalidad: escribir \textit{solo} el código necesario para hacer pasar el test
          (que salga en verde).
    \item Refactorizar: iterar sobre la solución obtenida para mejorarla.
\end{itemize}

\subsection{Github}

Para implementar lo enunciado anteriormente, el trabajo ha sido desarrollado usando \textit{git} y \textit{GitHub}.
Para ello, las tareas necesarias se han dividido en hitos \cite{milestones}. Cada hito contiene una
serie de \textit{issues}, o tareas a resolver, para alcanzar dicho hito.

Para implementar la tabla de Kanban también se ha usado Github, ya que implementa de forma nativa un
tablero que permite definir las columnas comunes en esta metodología:

\section{Usuarios}

Los usuarios o \textit{stakeholders} de este proyecto son:

\begin{enumerate}
      \item Tribunal: al ser un trabajo de fin de grado, el proyecto debe llevar asociado una memoria
            que explique el trabajo.
      \item Tutores: el trabajo esta cotutorizado por varias personas que no son expertas en ambas partes del trabajo.
      \item Científico de datos: investigador que quiera expandir este trabajo o usarlo como
            referencia en alguno de sus proyectos. Se asume que este usuario tiene alto conocimiento técnico.
      \item Guionista: profesional cuyo principal interés es modificar los tropos de su película para
            conseguir un buen rating. Se asume que este usuario no tiene, en principio, ningún conocimiento
            técnico.
\end{enumerate}

\section{Historias de usuario}

Una vez se conocen los usuarios del trabajo, se pueden desarrollar sus principales historias de usuario o casos de uso:

\subsection{[HU0] Configuración del proyecto} \label{uc:configuration}

Como tutores necesitamos disponer de una forma fácil de leer el resultado en cada
una de las iteraciones.

Como tribunal necesitamos disponer de una forma fácil de leer el resultado final.

\subsubsection{Criterios de aceptación}

\begin{itemize}
      \item Debe estar claro que metodología se está llevando a cabo para el desarrollo del trabajo.
      \item Se debe poder generar el trabajo de forma sencilla.
      \item Se debe incluir una sección en la que se detalle el estado del arte.
      \item Los objetivos e historias del usuario deben de estar definidos.
\end{itemize}

\subsection{[HU1] Análisis de tropos}

Como científico de datos necesito un modelo reproducible que pueda alimentar con mis propios
conjuntos de datos.

Como tribunal/tutor necesito ver el desarrollo matemático e informático necesario para realizar
un análisis semántico de los tropos para que pueda evaluar al estudiante.

En este primera iteración del trabajo los datos sobre \textit{rating} no se van a incluir para
hacer las historias de usuario más pequeñas e independientes (acorde a \textit{INVEST} \cite{buglione2013improving})

\subsubsection{Criterios de aceptación}

\begin{itemize}
      \item Dado un tropo, se debe poder obtener el más similar semánticamente.
      \item Dado un tropo, se deben poder ver los más \textit{cercanos} a él semánticamente.
      \item Dado un conjunto de tropos, se deben poder obtener sus representaciones para operar con ellos (\textit{embeddings}).
      \item El desarrollo del algoritmo tiene que estar claramente explicado, incluyendo el desarrollo de las herramientas matemáticas usadas.
\end{itemize}


\subsubsection{[HU2 ]Primera extensión del modelo Word2Vec}\label{uc:word2vec_primera_extension}

Como tribunal/tutor/científico de datos
necesito entender cómo se ha extendido el modelo Word2Vec para analizar la semántica de un conjunto de
tropos. Este extensión debe permitir que el modelo acepte conjuntos de palabras (tropos) sin orden, ya
los tropos son conjuntos no ordenados.

\subsubsection{Criterios de aceptación}

\begin{itemize}
      \item La implementación de la extensión debe estar documentada.
      \item El desarrollo del algoritmo tiene que estar claramente explicado, incluyendo el desarrollo de las herramientas matemáticas usadas.
\end{itemize}

\subsection{[HU3] Considerar datos sobre \textit{rating}}\label{uc:word2vec_rating}

Como guionista quiero poder asociar \textit{ratings} a los tropos para saber cómo van
a influir en mi siguiente obra.

Como tribunal/tutor/científico de datos necesito entender cómo se ha asociado la información
de \textit{ratings} al conjunto de tropos.

\subsubsection{Criterios de aceptación}

\begin{itemize}
      \item Dado un conjunto de tropos, se debe poder consultar información sobre el \textit{rating} que se obtendría.
      \item Dado un conjunto de tropos, se debe poder saber cómo modificar un tropo afectaría al rating.
      \item El desarrollo del algoritmo tiene que estar claramente explicado, incluyendo el desarrollo de las herramientas matemáticas usadas.
\end{itemize}

\section{Seguimiento del desarrollo}

La principal herramienta para el seguimiento del desarrollo ha sido GitHub, donde se ha creado \href{https://github.com/antoniogamiz/tfg}{un repositorio para el trabajo}.
Un histórico completo del trabajo, de forma incremental, puede ser consultado en la lista de \href{https://github.com/antoniogamiz/tfg/commits/main}{\textit{commits}}.
Además esto ha sido complementado con la herramienta de \textit{pull requests} de Github para poder hacer un desarrollo incremental de forma cómoda.

Además el trabajo se ha divido en hitos o \textit{milestones}, que son iteraciones incrementales y cada una de ellas es un producto mínimamente viable o MVP
(de \textit{Minimum Viable Product}):

\begin{itemize}
      \item \href{https://github.com/antoniogamiz/tfg/milestone/1}{Milestone 0 - Configuración inicial}: el objetivo es tener una forma fácil y reproducible de generar el trabajo,
      además de tener procesos para asegurar la calidad del trabajo (como correctores ortográficos cada vez que se hace un cambio).
      \item \href{https://github.com/antoniogamiz/tfg/milestone/4}{Milestone 1 - Teoría modelo \textit{Word2Vec}}: el objetivo es tener la fundación matemática del modelo usado desarrollada.
      Esa explicación también incluya la justificación de los algoritmos que el modelo usa.
      \item \href{https://github.com/antoniogamiz/tfg/milestone/6}{Milestone 2 - Implementación del modelo}: el objetivo es tener la implementación del modelo \textit{Word2Vec} y las pruebas
      necesarias para asegurarse que el modelo funciona correctamente. Además se incluirá la justificación de por qué se ha implementado el modelo desde cero, en lugar de usar
      una implementación existente.
      \item \href{https://github.com/antoniogamiz/tfg/milestone/7}{Milestone 3 - Extensión del modelo}: una vez se tiene el modelo inicial, el objetivo de este hito es modificar y extender la justificación
      matemática e implementación para que acepte tropos.
      \item \href{https://github.com/antoniogamiz/tfg/milestone/8}{Milestone 4 - Optimización de la implementación}: una vez el trabajo está completo, será necesario optimizar la implementación. Por lo que el
      objetivo de este hito es aplicar técnicas de optimización sobre la implementación.
\end{itemize}
