\chapter{Planificación}

\section{Metodología utilizada}


\section{Usuarios}

Los usuarios o \textit{stakeholders} de este proyecto son:

\begin{enumerate}
      \item Tribunal / Tutor: al ser un trabajo de fin de grado, el proyecto debe llevar asociado una memoria
            que explique el trabajo.
      \item Cientifíco de datos: investigador que quiera expandir este trabajo o usarlo como
            referencia en alguno de sus proyectos. Se asume que este usuario tiene alto conocimiento técnico.
      \item Guionista: profesional cuyo principal interés es modificar los tropos de su película para
            conseguir un buen rating. Se asume que este usuario no tiene, en principio, ningún conocimiento
            técnico.
      \item Estudiante: como estudiente de la carrera, necesito presentar una memoria donde explique el trabajo realizado.
\end{enumerate}

\section{Casos de uso}

Una vez se conocen los usuarios del trabajo, se pueden desarrollar sus principales casos de uso:

\subsubsection*{Configuración del proyecto} \label{uc:configuration}

Como tutor del trabajo necesito disponer de una forma fácil de leer el resultado final en cada
una de las iteraciones.

\subsubsection*{Datos de tropos}\label{uc:trope_data}: como desarrollador necesito obtener información acerca de los
tropos: a qué película pertenecen, cuántos hay, etc.

\subsubsection*{Datos de rating}\label{uc:rating_data}: como desarrollador necesito obtener información acerca del rating
de las películas.

\subsubsection*{Explicación del modelo Word2Vec}\label{uc:math}: como tutor/científico de datos/estudiante
necesito entender el desarrollo matemático del modelo Word2Vec.

\subsubsection*{Extensión del modelo Word2Vec}\label{uc:math}: como tutor/científico de datos/estudiante
necesito entender la extensión del modelo Word2Vec.

\subsubsection*{Herramienta fácil de usar}\label{uc:user_friendly_tool}: como guionista, necesito una herramienta
fácil de usar que no requiera conocimientos técnicos para ser usada.

\section{Temporización}

\section{Seguimiento del desarrollo}
