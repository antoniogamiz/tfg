\chapter{Planificación}

En este capítulo se detalla la planificación del proyecto y las metodologías y buenas prácticas
usadas para el desarrollo del mismo. Siguiendo uno los objetivos de este trabajo, la planificación
ha sido muy influenciada por el desarrollo ágil aplicado a la ciencia \cite{desarrolloagil}.

\section{Metodología utilizada}

\subsection{Kanban}

Como el objetivo del trabajo es \textit{ser ágil}, es necesario escoger una metodología de trabajo que
lo permita. En este proyecto se ha escogido \textit{Kanban}. Los objetivos principales de kanban son dar
visibilidad al trabajo que se está haciendo, maximizar la eficiencia y limitar la cantidad de trabajo
en progreso. Esto se hace a través de un tablero en el que de un simple vistazo, se puede consultar el
estado del trabajo. Esto es realmente útil para el objetivo \ref{obj:3}, ya que permite a los tutores
ver en todo momento en qué estado se encuentra el proyecto.

Kanban se base en hacer pequeños cambios incrementales de forma que el esfuerzo para que se puede realizar
la tarea sea mínimo. Esto no implica que Kanban favorezca la rapidez sobre la calidad, al contrario,
prevalece la calidad del producto final sobre la velocidad de desarrollo.

\subsection{Desarrollo dirigido por pruebas}

Kanban es una metodología para organizar el trabajo. A la hora de escribir código, es necesario usar
buenas prácticas adicionales. El desarrollo dirigido por tests (TDD, del inglés \textit{Test Driven Development})
es un gran complemento a Kanban, ya que nos permite hacer incrementos pequeños que aporten valor al
producto final. El flujo de trabajo consiste en:

\begin{itemize}
    \item Escribir test: definir un test que comprueba que la nueva funcionalidad funciona como es debido.
          Evidentemente, al principio debe fallar (salir en rojo) porque no hay nada implementado.
    \item Implementar funcionalidad: escribir \textit{solo} el código necesario para hacer pasar el test
          (que salga en verde).
    \item Refactorizar: iterar sobre la solución obtenida para mejorarla.
\end{itemize}

\subsection{Github}

Para implementar lo enunciado anteriormente, el trabajo ha sido desarrollado usando \textit{git} y \textit{GitHub}.
Para ello, las tareas necesarias se han dividido en hitos \cite{milestones}. Cada hito contiene una
serie de \textit{issues}, o tareas a resolver, para alcanzar dicho hito.

Para implementar la tabla de Kanban también se ha usado Github, ya que implementa de forma nativa un
tablero que permite definir las columnas comunes en esta metodología:

\begin{itemize}
    \item Tareas por hacer
    \item En progreso
    \item Hechas
\end{itemize}

Además esto ha sido complementado con la herramienta de \textit{pull requests} de Github para poder hacer
un desarrollo incremental de forma cómoda.
