\chapter{Planificación}

\section{Metodología utilizada}


\section{Usuarios}

Los usuarios o \textit{stakeholders} de este proyecto son:

\begin{enumerate}
      \item Tribunal: al ser un trabajo de fin de grado, el proyecto debe llevar asociado una memoria
            que explique el trabajo.
      \item Tutores: el trabajo esta cotutorizado por varias personas que no son expertas en ambas partes del trabajo.
      \item Cientifíco de datos: investigador que quiera expandir este trabajo o usarlo como
            referencia en alguno de sus proyectos. Se asume que este usuario tiene alto conocimiento técnico.
      \item Guionista: profesional cuyo principal interés es modificar los tropos de su película para
            conseguir un buen rating. Se asume que este usuario no tiene, en principio, ningún conocimiento
            técnico.
\end{enumerate}

\section{Casos de uso}

Una vez se conocen los usuarios del trabajo, se pueden desarrollar sus principales casos de uso:

\subsubsection*{Configuración del proyecto} \label{uc:configuration}

Como tutores necesitamos disponer de una forma fácil de leer el resultado en cada
una de las iteraciones.

Como tribunal necesitamos disponer de una forma fácil de leer el resultado final.

\subsubsection*{Datos de tropos}\label{uc:trope_data}

Como desarrollador necesito obtener información acerca de los
tropos: a qué película pertenecen, cuántos hay, etc.

\subsubsection*{Datos de rating}\label{uc:rating_data}

Como desarrollador necesito obtener información acerca del rating
de las películas.

\subsubsection*{Explicación del modelo Word2Vec}\label{uc:math}

Como tribunal/tutor/científico de datos/estudiante
necesito entender el desarrollo matemático del modelo Word2Vec.

\subsubsection*{Extensión del modelo Word2Vec}\label{uc:math_extension}

Como tribunal/tutor/científico de datos/estudiante
necesito entender la extensión del modelo Word2Vec.

\subsubsection*{Herramienta fácil de usar}\label{uc:user_friendly_tool}

Como guionista, necesito una herramienta
fácil de usar que no requiera conocimientos técnicos para ser usada.

\section{Temporización}

\section{Seguimiento del desarrollo}
