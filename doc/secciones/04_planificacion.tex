\chapter{Planificación}

\section{Metodología utilizada}


\section{Casos de uso}

Los usuarios o \textit{stakeholders} de este proyecto son:

\begin{enumerate}
      \item Tribunal / Tutor: al ser un trabajo de fin de grado, el proyecto debe llevar asociado una memoria
            que explique el trabajo.
      \item Cientifíco de datos: investigador que quiera expandir este trabajo o usarlo como
            referencia en alguno de sus proyectos. Se asume que este usuario tiene alto conocimiento técnico.
      \item Guionista: profesional cuyo principal interés es modificar los tropos de su película para
            conseguir un buen rating. Se asume que este usuario no tiene, en principio, ningún conocimiento
            técnico.
      \item Estudiante: como estudiente de la carrera, necesito presentar una memoria donde explique el trabajo realizado.
\end{enumerate}

Una vez se conocen los usuarios del trabajo, se pueden desarrollar sus principales casos de uso:

\begin{enumerate}
      \item \label{uc:configuration} Configuración del proyecto \cite{configuration_milestone}: como
            tutor del trabajo necesito disponer de una forma fácil de leer el resultado final en cada
            una de las iteraciones.
      \item \label{uc:trope_data} Datos de tropos: como desarrollador necesito obtener información acerca de los
            tropos: a qué película pertenecen, cuántos hay, etc.
      \item \label{uc:rating_data} Datos de rating: como desarrollador necesito obtener información acerca del rating
            de las películas.
      \item \label{uc:math} Matemáticas: como tutor/científico de datos necesito poder consultar el desarrollo
            matemático de la solución implementada.
      \item \label{uc:user_friendly_tool} Herramienta fácil de usar: como guionista, necesito una herramienta
            fácil de usar que no requiera conocimientos técnicos para ser usada.
\end{enumerate}

\section{Temporización}

\section{Seguimiento del desarrollo}
