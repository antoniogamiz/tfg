\chapter{Estado del arte}

El concepto de \emph{tropo} para el cine fue definido por Michael Rizzo \cite{rizzo2013art} en el año 2013. Este término
tiene el mismo origen que los tropos para la literatura, cuyo estudio y clasificación fue una importante investigación en el
campo de la retórica (como por ejemplo, el estudio de tropos en películas \cite{nelson1998tropes}). En el ámbito del cine,
los tropos han sido usados para el estudio de la semiología de las películas \cite{howtoreadafilm}, que incluye el estudio
de los signos o elementos de una película y su significado (semántica).

Hay algunos estudios recientes sobre los tropos en películas \cite{garcia2018overview} que han ayudado a interpretar las
películas y los tropos que se encuentran dentro de ellas. En este trabajo se pretende extender ese estudio aplicando el uso
de \emph{embeddings}, para extraer más información sobre los tropos y su semántica.

La primera noción de \textit{embedding} proviene del concepto matemático de una instancia de alguna
estructura matemática contenida en otra. Como por ejemplo, los números naturales, contenidos en los números enteros
$\mathbb{N}$. Seguramente inspirado por este concepto, surgió el concepto de \textit{embedding} en la rama del procesamiento
del lenguaje natural.

A lo largo del tiempo, numerosos modelos han sido usados para encontrar o calcular estos \textit{embeddings}, cada uno
con una aproximación diferente. A continuación se ven algunos de estos modelos.

En el modelo \emph{word2vec} \cite{word2vec:1} \cite{word2vec:2}, se crea una representación
en forma de vector para cada palabra. El principal problema de este modelo es que cada palabra tiene un único vector asociado,
pero en el mundo real, cada palabra puede tener distintos significados dependiendo del contexto.
Por ejemplo, en las frases \emph{kiwi y mango son coches españoles}, y \emph{kiwi y mango son frutas},
el mismo word embedding es usado para kiwi y mango.

Modelos como \emph{Adaptive Skipgram} \cite{adaptiveskipgram}, intentan solventar este problema usando
varios vectores para \emph{cada} sentido de una palabra, generando así un vocabulario de sentidos de palabras.
Sin embargo, se sigue usando el mismo embedding para cada ocurrencia de un sentido en específico.

Entre los últimos avances se encuentra \emph{BERT/ELMO} \cite{bert}, un exitoso modelo que soluciona el
problema antes descrito. En lugar de aportar información sobre el significado de una palabra, se construye
un embedding dependiente del contexto (y por lo tanto, específico a cada aparición). De forma que la palabra
\emph{kiwi} tendrá diferentes embeddings en las frases \emph{kiwi y mango son coches españoles}, y
\emph{kiwi y mango son frutas}.

En el caso de este trabajo, los tropos no tienen, a priori, distintos sentidos, por lo que se ha optado
por usar el modelo \emph{word2vec}.

Como el nombre del trabajo indica, la intención es generar \textit{any embeddings}, es decir, no solo aplicado a
palabras \cite{word2vec:1} \cite{word2vec:2} o frases \cite{kiros2015skip}. En este trabajo en particular, se va a centrar
el estudio en generar embeddings para tropos de películas.
